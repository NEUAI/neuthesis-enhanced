\chapter{LaTeX使用说明}\label{chap:guide}

为方便使用及更好地展示LaTeX排版的优秀特性,neuthesis的框架和文件体系进行了细致地处理,尽可能地对各个功能和板块进行了模块化和封装,对于初学者来说,众多的文件目录也许一开始让人觉得有些无所适从,但阅读完下面的使用说明后,会发现原来使用思路是简单而清晰的,而且,当对LaTeX有一定的认识和了解后,会发现其相对Word类排版系统极具吸引力的优秀特性。所以,如果是初学者,请不要退缩,请稍加尝试和坚持,以领略到LaTeX的非凡魅力,并可以通过阅读相关资料如LaTeX Wikibook\cite{wikibook2014latex}来完善自己的使用知识。

\section{先试试效果}

\begin{enumerate}
    \item 安装软件:根据所用操作系统和章节~`\ref{sec:system}`中的信息安装LaTeX编译环境。
    \item 获取模板:下载 \href{https://github.com/mervin0502/neuthesis}{neuthesis} 模板并解压。neuthesis模板不仅提供了相应的类文件,同时也提供了包括参考文献等在内的完成学位论文的一切要素,所以,下载时,推荐下载整个neuthesis文件夹,而不是单独的文档类。
    \item 编译模板:
        \begin{enumerate}
            \item Windows:双击运行artratex.bat脚本。
            \item Linux或MacOS: {\scriptsize \verb|terminal| -> \verb|chmod +x ./artratex.sh| -> \verb|./artratex.sh xa|}
            \item 任意系统:都可使用LaTeX编辑器打开Thesis.tex文件并选择xelatex编译引擎进行编译。
        \end{enumerate}
    \item 错误处理:若编译中遇到了问题,请先查看“常见问题”(章节~\ref{sec:qa})。
\end{enumerate}

编译完成即可获得本PDF说明文档。而这也完成了学习使用neuthesis撰写论文的一半进程。什么?这就学成一半了,这么简单???,是的,就这么简单!

\section{文档目录简介}

\subsection{Thesis.tex}

Thesis.tex为主文档,其设计和规划了论文的整体框架,通过对其的阅读可以了解整个论文框架的搭建。

\subsection{编译脚本}

\begin{itemize}
    \item Windows:双击Dos脚本artratex.bat可得全编译后的PDF文档,其存在是为了帮助不了解LaTeX编译过程的初学者跨过编译这第一道坎,请勿通过邮件传播和接收此脚本,以防范Dos脚本的潜在风险。
    \item Linux或MacOS:在terminal中运行
        \begin{itemize}
            \item \verb|./artratex.sh xa|:获得全编译后的PDF文档
            \item \verb|./artratex.sh x|:快速编译模式
        \end{itemize}
    \item 全编译指运行 \verb|xelatex+bibtex+xelatex+xelatex| 以正确生成所有的引用链接,如目录,参考文献及引用等。在写作过程中若无添加新的引用,则可用快速编译,即只运行一遍LaTeX编译引擎以减少编译时间。
\end{itemize}

\subsection{Tmp文件夹}

运行编译脚本后,编译所生成的文档皆存于Tmp文件夹内,包括编译得到的PDF文档,其存在是为了保持工作空间的整洁,因为好的心情是很重要的。

\subsection{Style文件夹}

包含neuthesis文档类的定义文件和配置文件,通过对它们的修改可以实现特定的模版设定。若需更新模板,一般只需用新的样式文件替换旧的即可。

\begin{enumerate}
    \item neuthesis.cls:文档类定义文件,论文的最核心的格式即通过它来定义的。
    \item neuthesis.cfg:文档类配置文件,设定如目录显示为“目~录”而非“目录”。
    \item artratex.sty: 常用宏包及文档设定,如参考文献样式、文献引用样式、页眉页脚设定等。这些功能具有开关选项,常只需在Thesis.tex中的如下命令中进行启用即可,一般无需修改artratex.sty本身。
        
        \path{\usepackage[options]{artratex}} 
    \item artracom.sty:自定义命令以及添加宏包的推荐放置位置。
\end{enumerate}

\subsection{Tex文件夹}

文件夹内为论文的所有实体内容,正常情况下,这也是\textbf{使用neuthesis撰写学文论文时,主要关注和修改的一个位置,注:所有文件都必须采用UTF-8编码,否则编译后将出现乱码文本},详细分类介绍如下:

\begin{itemize}
    \item Frontpage.tex:为论文中英文封面及中英文摘要。\textbf{论文封面会根据英文学位名称如Bachelor,Master,或是Doctor自动切换为相应的格式}。
    \item Mainmatter.tex:索引需要出现的Chapter。开始写论文时,可以只索引当前章节,以快速编译查看,当论文完成后,再对所有章节进行索引即可。
    \item Chap{\_}xxx.tex:为论文主体的各个章节,可根据需要添加和撰写。
    \item Appendix.tex:为附录内容
    \item Backmatter.tex:为发表文章信息和致谢部分等。
\end{itemize}

\subsection{Img文件夹}

用于放置论文中所需要的图类文件,支持格式有:.jpg, .png, .pdf。其中,\verb|neu_logo.pdf|为东北大学校徽。不建议为各章节图片建子目录,即使图片众多,若命名规则合理,图片查询亦是十分方便。

\subsection{Biblio文件夹}

\begin{enumerate}
    \item ref.bib:参考文献信息库。
    \item gbt7714-xxx.bst:符合国标的文献样式定义文件。由 \href{https://github.com/zepinglee/gbt7714-bibtex-style}{zepinglee}  开发,并满足最新国标要求。与文献样式有关的问题,请查阅开发者所提供的文档,并建议适当追踪其更新。
\end{enumerate}

\section{常见使用问题}\label{sec:qa}

\begin{enumerate}
    \item 模板每次发布前,都已在Windows,Linux,MacOS系统上测试通过。下载模板后,若编译出现错误,则请见 \href{https://github.com/mervin0502/neuthesis/wiki}{neuthesis和LaTeX知识小站} 的 \href{https://github.com/mervin0502/neuthesis/wiki/%E7%BC%96%E8%AF%91%E6%8C%87%E5%8D%97}{编译指南}。

    \item 模板文档的编码为UTF-8编码。所有文件都必须采用UTF-8编码,否则编译后生成的文档将出现乱码文本。若出现文本编辑器无法打开文档或打开文档乱码的问题,请检查编辑器对UTF-8编码的支持。如果使用WinEdt作为文本编辑器(\textbf{不推荐使用}),应在其Options -> Preferences -> wrapping选项卡下将两种Wrapping Modes中的内容:
        
        TeX;HTML;ANSI;ASCII|DTX...
        
        修改为:TeX;\textbf{UTF-8|ACP;}HTML;ANSI;ASCII|DTX...
        
        同时,取消Options -> Preferences -> Unicode中的Enable ANSI Format。

    \item 推荐选择xelatex或lualatex编译引擎编译中文文档。编译脚本的默认设定为xelatex编译引擎。你也可以选择不使用脚本编译,如直接使用 LaTeX文本编辑器编译。注:LaTeX文本编辑器编译的默认设定为pdflatex编译引擎,若选择xelatex或lualatex编译引擎,请进入下拉菜单选择。为正确生成引用链接,需要进行全编译。

    \item Texmaker使用简介
        \begin{enumerate}
            \footnotesize
            \item 使用 Texmaker “打开 (Open)” Thesis.tex。
            \item 菜单 “选项 (Options)” -> “设置当前文档为主文档 (Define as Master Document)”
            \item 菜单 “自定义 (User)” -> “自定义命令 (User Commands)” -> “编辑自定义命令 (Edit User Commands)” -> 左侧选择 “command 1”,右侧 “菜单项 (Menu Item)” 填入 Auto Build -> 点击下方“向导 (Wizard)” -> “添加 (Add)”: xelatex + bibtex + xelatex + xelatex + pdf viewer -> 点击“完成 (OK)”
            \item 使用 Auto Build 编译带有未生成引用链接的源文件,可以仅使用 xelatex 编译带有已经正确生成引用链接的源文件。
            \item 编译完成,“查看(View)” PDF,在PDF中 “ctrl+click” 可链接到相对应的源文件。
        \end{enumerate}
    
    \item 模版的设计可能地考虑了适应性。致谢等所有条目都是通过最为通用的

        \verb+\chapter{item name}+  and \verb+\section*{item name}+

        来显式实现的 (请观察Backmatter.tex),从而可以随意添加,放置,和修改,如同一般章节。对于图表目录名称则可在neuthesis.cfg中进行修改。

    \item 设置文档样式: 在artratex.sty中搜索关键字定位相应命令,然后修改
        \begin{enumerate}
            \item 正文行距:启用和设置 \verb|\linespread{1.5}|,默认1.5倍行距。
            \item 参考文献行距:修改 \verb|\setlength{\bibsep}{0.0ex}|
            \item 目录显示级数:修改 \verb|\setcounter{tocdepth}{2}|
            \item 文档超链接的颜色及其显示:修改 \verb|\hypersetup|
        \end{enumerate}

    \item 文档内字体切换方法:
        \begin{itemize}
            \item 宋体:东北大学论文模板neuthesis 或 \textrm{东北大学论文模板neuthesis}
            \item 粗宋体:{\bfseries 东北大学论文模板neuthesis} 或 \textbf{东北大学论文模板neuthesis}
            \item 黑体:{\sffamily 东北大学论文模板neuthesis} 或 \textsf{东北大学论文模板neuthesis}
            \item 粗黑体:{\bfseries\sffamily 东北大学论文模板neuthesis} 或 \textsf{\bfseries 东北大学论文模板neuthesis}
            \item 仿宋:{\ttfamily 东北大学论文模板neuthesis} 或 \texttt{东北大学论文模板neuthesis}
            \item 粗仿宋:{\bfseries\ttfamily 东北大学论文模板neuthesis} 或 \texttt{\bfseries 东北大学论文模板neuthesis}
            \item 楷体:{\itshape 东北大学论文模板neuthesis} 或 \textit{东北大学论文模板neuthesis}
            \item 粗楷体:{\bfseries\itshape 东北大学论文模板neuthesis} 或 \textit{\bfseries 东北大学论文模板neuthesis}
        \end{itemize}

    \item 封面下划线上的文本不居中下划线,这是因为下划线前面还有字头,导致文本只能在页面居中和在下划线上居中二选一。当前封面采取页面居中。如需要调整文本在下划线上的位置,可用 \verb|\hspace{+/- n.0em}| 命令来插入或删除 n 个空格,进行手动调整,比如

        \verb|\advisor{\hspace{+3.0em} xxx~研究员~xxx单位}|
                
    有时下划线看上去粗细不一致,这是显示的问题,打印正常。
\end{enumerate}


